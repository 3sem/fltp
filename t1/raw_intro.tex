 Назовём множество абстрактных объектов -- символов -- алфавитом $\Sigma$. Пусть алфавит конечный. Пустой и бесконечный алфавиты нам неинтересны.

Введём слово над алфавитом $\Sigma: w(A)={a_i, a_i \in \Sigma, \forall i=0..|w(A)|}$ -- последовательность символов из алфавита $, 0 =< |w(\Sigma)| <+\infty$.

Чтобы оперировать словами длины 0, вводят специальный символ длины 0 -- $\epsilon:|\epsilon^n|=0, n=0..+\infty$; Его называют пустым.

Обозначим множество таких последовательностей из символов алфавита $\Sigma$, включая слово длины 0, как $\Sigma^*$.Тогда некоторый язык $L(\Sigma)$ над алфавитом $\Sigma$ можно задать как подмножество слов над алфавитом: $\L(\Sigma)\subset(\Sigma^*)$. Таким образом, математически мы определили объекты, с которыми будем работать, это последовательности конечной длины и множества.

Теория формальных языков по сути -- математический способ конструктивного описания множеств последовательностей (слов) элементов некоторых множеств (алфавитов). Почему конструктивного? Потому что, в принципе, все слова языка можно просто перечислить, если:
\begin{enumerate}
\item любое слово -- конечной длины.
\item множество слов конечно.
\item нет ограничений на временную сложность алгоритмов, используемых в работе с таким языком.
\end{enumerate}
1) и 2), соответственно, говорят о том, что мы можем перечислять слова бесконечно, что теряет смысл. 3) это больше практическая хотелка -- нам нужны алгоритмы, которые работают, по крайней мере, за полином небольшой степени и по времени, и по памяти, так как мы хотим работать с относительно мощными языками, и нам важна масштабируемость.

В нашем курсе 1) будет всегда выполняться: считаем, что любое слово языка -- конечной длины. Но пусть 2) и 3) не выполняются. Тогда задача конструктивного, то есть 'сжатого' и точного описания множества слов обретает куда более глубокую практическую значимость.

Кроме перечисления, можно предложить еще 2 способа задания языка:
\begin{itemize}
\item Формальный вычислитель -- все слова языка можно распознать некоторой вычислительной машиной.
\item Генератор -- все слова языка можно вывести посредством формальной процедуры переписывания строк по системе правил. Система математических объектов, позволяющих это сделать, называется формальной грамматикой.
\end{itemize}
С этими двумя способами теория формальных языков и работает.

Мы начнём с первого, в последствии переключимся на второй, а затем синхронно двинемся дальше с обеими способами, усложняя и рассматриваемые методы, подходы и задачи.
